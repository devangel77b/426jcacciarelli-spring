\documentclass[12pt,conference,onecolumn]{IEEEtran}

\usepackage[hidelinks]{hyperref}

\title{Study on methods of memorization}
\author{%
\IEEEauthorblockN{Jake Cacciarelli}\IEEEauthorblockA{Science \& Engineering\\Manalapan High School\\Englishtown, NJ\\\href{mailto:426jcacciarelli@frhsd.com}{426jcacciarelli@frhsd.com}}\and
\IEEEauthorblockN{Kriti Malhotra}\IEEEauthorblockA{Science \& Engineering\\Manalapan High School\\Englishtown, NJ\\\href{mailto:426kmalhotra@frhsd.com}{426kmalhotra@frhsd.com}}\and
\IEEEauthorblockN{Samay Prabhu}\IEEEauthorblockA{Science \& Engineering\\Manalapan High School\\Englishtown, NJ\\\href{mailto:426sprabhu@frhsd.com}{426sprabhu@frhsd.com}}}

\date{June 16, 2026}

\newcommand{\keywords}{psychology, memory, recall, association}

\usepackage{hyperref}
\makeatletter
\AtBeginDocument{
\hypersetup{%
pdftitle={\@title},
pdfauthor={Jake Cacciarelli, Kriti Malhotra, and Samay Prabhu},
pdfkeywords={\keywords}}}
\makeatother

\begin{document}
\maketitle 

\begin{abstract}
Memory is formed in the hippocampus of our brains and is something that has been researched over the past few decades. Many different encoding and retrieval methods can impact the way our memories are processed and stored in our long-term memory. This study will evaluate the effectiveness of several strategies in order to enhance both short-term and long-term memory. We will test techniques such as spacing effect, Method of Loci, and the visuospatial workspace as well as mnemonic devices and structural analysis, to study the impact of memory retention when using both deep and shallow processing. Collecting a random sample of Manalapan High School seniors, we will identify differences in performance across various conditions, which will allow us to find the most effective procedure for memory retention in high school students.
\end{abstract}

\begin{IEEEkeywords}
\keywords
\end{IEEEkeywords}

\end{document}
